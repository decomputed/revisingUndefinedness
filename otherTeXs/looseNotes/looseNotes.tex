\documentclass[a4paper,11pt]{article}

\usepackage[latin1]{inputenc}
\usepackage{graphicx}
\usepackage{url}
\usepackage{amssymb}
\usepackage{xspace}
\usepackage{amsmath}


\setlength{
\parskip}{5mm}
\setlength{
\parindent}{0mm}

\usepackage{abbrev}

\usepackage{xyling}




\newtheorem{definition}{Definition}
\newtheorem{example}{Example}





%%%%%%%%%%%%%%%%%%%%%%%%%%%%%%%%%%%%%%%%%%%
%
%   document starts here
%

\begin{document}

\title{Loose notes and conclusions\\
on a \fpo for rSM and rWFS}
\author{-- DRAFT VERSION --}
\date{Lu�s Rodrigues Soares\\
March 27th, 2006}



\maketitle


%%%%%%%%%%%%%%%%%%%%%%%%%%%%%%%%%%%%%%%%%%%%%%%%%%%%  SECTION
\section*{Introductory notes} This document includes a set of conclusions and steps taken in the direction of defining a
\fpo for rSMs and rWFS. It is an account of the past 4 months of work and includes all ideas and intuitions tested so
far. Its tone is rather informal and I'll be introducing all necessary concepts as they are required.


\newpage
\part{Four Months of \CFPs} I start by describing the last four months of definitions based on generating \cfps and
calculating the stable models of the CFPs along with the original program.

%%%%%%%%%%%%%%%%%%%%%%%%%%%%%%%%%%%%%%%%%%%%%%%%%%%%  SECTION
\section{The first definition\protect\footnote{Cova do Vapor version}} We were somewhere in December when the first
stable idea for a \fpo came. Professor Moniz Pereira talked to me and Alexandre of the idea of \CFPs (CFPs) and how they
were used to help prove a certain literal in a RAA chain. This version of the definition of CFPs was unindexed in the
default literals and throughout the rest of the document I'll refer to it as the unindexed version:



%###########################################            DEFINITION
\begin{definition}[Unindexed \CFP $P_{cf}$] Let $P$ be a \nlp with $l$ rules in the form $r_k=H\leftarrow
A_{1},\ldots,A_{m},\sim B_{1},\ldots,\sim B_{n}$ ($m,n\geq 0\ and\ 1\leq k\leq l$) and $X$ a literal such that

\begin{align*}
\exists_{r_{k} \in P}:&X\in \left\{B_{1},\ldots,B_{n}\right\}
\end{align*}

The \cfp $P_{cf}\left(X\right)$ regarding $X$ and rule $r_{k}$ in program $P$ is the following transformation of $P$:

\begin{enumerate}
\item Replace all non-negated literals $A$ in all rules with the auxiliary literal $A_{\iota}$;
\item Remove from $r_{k}$ all default literals $\sim X$;
\item Add to $P_{cf}\left(X_\iota\right)$ the rule $X\leftarrow X_{\iota}$
\end{enumerate}

The index used ($\iota$, iota) is unique for this \cfp and literal $X$ and is called the \cf index of
$P_{cf}\left(X\right)$ and it identifies the literal we are trying to conclude.
\end{definition}
%###########################################            END   DEFINITION


The idea was that in order to prove a certain literal $X$ by \raa (RAA) one would have to use a couterfactual program
where we would assume $\sim X$. For example, consider a direct odd loop:

\begin{align*} P=\{&a\leftarrow x\\
&x\leftarrow y\\
&y\leftarrow\sim a\}
\end{align*}

The CFP for $a$ would be:

\begin{align*} P_{cf}\left(a\right)=\{&a\leftarrow a_{1}\\
&a_{1}\leftarrow x_{1}\\
&x_{1}\leftarrow y_{1}\\
&y_{1}\leftarrow\}
\end{align*}


In this program, we assume $\sim a$ as being true and try to see if $a$ is a model of the program. We changed all
literals to their indexed versions so we can later remove them from the program, i.e., the indexed literals are only
necessary to create the RAA chain between $\sim a$ and $a$. The only thing provable from the CFP is actually $a$,
because all indexed literals are not to be considered.

So, now we have:

\begin{align*} 
&\Gamma^r_{P}\left(I\right)=\\
=&\Gamma_{P_{cf}\left(a\right)\cup P}\left(I\right)=\\
=&\left\{a_1,x_1,y_1,a\right\}=\\
=&\left\{a\right\}
\end{align*}

So the idea is that we should be able to use thew CFPs of the atoms in an interpretation $I$, along with the original
one, and if $I$ was a stable model (SM) of $P\cup P_{cf}s$, then $I$ would be a rSM. We could then remove the indexed
literals and only the unindexed ones were used in the result. Problems started precisely because when we are considering
several literals in an interpretation they may interfere with each other's CFPs. For the odd loop of 3, $I=\{a,b\}$ is
not a SM of $P\cup P_{cf}\left(a\right)\cup P_{cf}\left(b\right)$:

\begin{align*} P=\{&a\leftarrow\sim b\\
&b\leftarrow\sim c\\
&c\leftarrow\sim a\}
\end{align*}

the CFPs for $a$ and $b$ would be:

\begin{align*} P_{cf}\left(a\right)=\{&a\leftarrow a_{1} & P_{cf}\left(b\right)=\{&b\leftarrow b_{2}\\
&a_{1}\leftarrow\sim b & &a_{2}\leftarrow\\
&b_{1}\leftarrow\sim c & &b_{2}\leftarrow\sim c\\
&c_{1}\leftarrow\} & &c_{2}\leftarrow\sim a\}
\end{align*}

However,  

\begin{align*} 
&\Gamma_{P_{cf}\left(a\right)\cup P_{cf}\left(b\right)\cup P}\left(I\right)=\\
=&\left\{b,b_1,c_1,a_2,b_2\right\}=\\
=&\left\{b\right\}
\end{align*}

From this observation, we came up with the idea that $a_{1}$ should be able to derive its truth value from $a_{2}$, sort
of like an auxiliary assumption, in order to give context to the fact that we were using several CFPs together.
Simmetrically, $b_{2}$ could also depend on $b_{1}$. Auxiliary assumptions undergone several versions, as they were more
or less permissive and more or less \emph{hacked}.


%########################################### DEFINITION
\begin{definition}[Permissive \CFA Assumptions operator $\alpha_{cf}$]

Let $P$ be a \nlp, $I=\left\{I_{1},\ldots,I_{j}\right\}$ a set of literals such that $1\leq i\leq j$ and $B$ a set of
\cfps obtained from the generation of a \cfp for each literal in $I$.

For each \cfp $P_{cf}\left(X\right)$ regarding a literal $X$ that contains a rule in the form:

\begin{center}
\begin{math} H_{\iota}\leftarrow A_{1},\ldots,A_{m},\sim B_{1},\ldots,\sim B_{n}
\end{math}
\end{center}

such that 

\begin{align*}
\exists_{P_{cf}\left(H\right)}:the\ counter\ factual\ index\ for\ P_{cf}\left(H\right)\ is\ \kappa
\end{align*}

introduce the rule $H_{\iota}\leftarrow H_{\kappa}$

\end{definition}%###########################################            END   DEFINITION

Roughly speaking, this would give us the new result for $P_{cf}\left(a\right)$ and $P_{cf}\left(b\right)$:

\begin{align*} 
P_cf\left(a\right)=\{a&\leftarrow a_1 & P_cf\left(b\right)=\{ b&\leftarrow b_2\\
a_1&\leftarrow\sim b & a_2&\leftarrow\\
b_1&\leftarrow\sim c & b_2&\leftarrow\sim c\\
c_1&\leftarrow & c_2&\leftarrow\sim a\\
a_1&\leftarrow a_2\} & b_2&\leftarrow b_1\}
\end{align*}

now giving us the intended result

\begin{align*} &\Gamma_{P_{cf}\left(a\right)\cup P_{cf}\left(b\right)\cup P}\left(I\right)=\\
=&\left\{b,b_1,c_1,a_1,a_2,b_2\right\}=\\
=&\left\{a,b\right\}
\end{align*}

However, if $I=\{a,b,c\}$, which is not a revised stable model, our CFPs would be

\begin{align*} 
P_cf\left(a\right)=\{a&\leftarrow a_1 & P_cf\left(b\right)=\{ b&\leftarrow b_2 & P_cf\left(c\right)=\{ c&\leftarrow
c_3\\
a_1&\leftarrow\sim b & a_2&\leftarrow & a_3&\leftarrow\sim b\\
b_1&\leftarrow\sim c & b_2&\leftarrow\sim c & b_3&\leftarrow\\
c_1&\leftarrow & c_2&\leftarrow\sim a & c_3&\leftarrow\sim a\\
a_1&\leftarrow a_3 & b_2&\leftarrow b_1 & c_3&\leftarrow c_1\\
a_1&\leftarrow a_2\} & b_2&\leftarrow b_3\} & c_3&\leftarrow c_2\}
\end{align*}

thus allowing $\Gamma_{P_{cf}\left(a\right)\cup P_{cf}\left(b\right)\cup P_{cf}\left(c\right)\cup
P}\left(I\right)=\{a,b,c\}$. 

So apparently, our auxiliary assumptions have to be more restrictive. Our next definition of the auxiliary assumptions
added some defaults to each added rule; more specifically we would add to each auxiliary rule the defaults present in
the original rule in program $P$, regarding the literal we were considering. For $I=\{a,b\}$, $P_{cf}\left(a\right)$
would become:

\begin{align*} 
P_cf\left(a\right)=\{a&\leftarrow a_1\\
a_1&\leftarrow\sim b\\
b_1&\leftarrow\sim c\\
c_1&\leftarrow\\
a_1&\leftarrow a_2,\sim c\}
\end{align*}

because as $a_2$ was present in $P_{cf}\left(b\right)$, it would have to respect the original defaults of the rule for
head $b$ in the original program\footnote{The definition is very unintuitive and there is no point in presenting it here
because it added no interesting notion to our work.}.

This is the definition I brought with me to the first exam of KRR, before professor Moniz Pereira went to Indonesia. It
became clear, during the exam, that the definition was very dificult to explain, and the intuitions were not very clear.
Afterwards, I noticed that all the ideas I had presented still didn't work in some programs, which eventually took me to
the next definition. 

\section{Afternoons at Praia da Torre} It was with the following program that I eventually came up with the most
promising definition so far. The program that originated this definition was:

\begin{align*} P=\{&a\leftarrow\sim b\\
&b\leftarrow\sim c, x\\
&c\leftarrow\sim a\\
&x\leftarrow\sim x,a\}
\end{align*}

I'll skip all the steps I eventually went through, just to enumerate all the tricks I added in this definition:

\begin{description}
\item[Direct OLONs] All direct single OLONs $\left(a\leftarrow\sim a\right)$ were solved before anuthing else was done.
This was because we already knew how these loops were handled, what their result was. Basically, if all our problems
were like this, we would simply add a rule to the transformation which stated somthing like ``remove $\sim X$ from the
body of all rules with head $X$''
\item[Several CF programs for the same CF assumption] If several $\sim X$ were present in the program, we would be
creating a CFP for each on of them. This was because we viewed each $\sim X$ as one possible counterfactual path to $X$,
so we would treat each path differently (we'll see later that this idea was indeed wrong).
\item[Facts] Additionally, we would be not create CFPs for literal which were facts in the program. Although it was a
trick, which if it wasn't present some programs would be giving more results than they should give, I still maintain
that CFPs should not be created in this condition.
\item[Auxiliary assumptions] Finally, and probably most importantly, the auxiliary assumptions were now a lot simpler:
$a_1$ could simply depend on $b_2$ if the rule for $a_1$ originally depended on $\sim b$, an idea of professor Moniz
Pereira.
\end{description}

This is where the most promissing definition of \gr was born. After some comments on the second exam of KRR I divided it
in the following definitions :

%########################################### DEFINITION
\begin{definition}[\CFP $P_{cf}$]

Let $P$ be a \nlp with $l$ rules in the form $r_k=H\leftarrow A_{1},\ldots,A_{m},\sim B_{1},\ldots,\sim B_{n}$ ($m,n\geq
0\ and\ 1\leq k\leq l$) and $X$ a literal such that


\begin{align*}
\nexists_{r_k \in P}:&r_k=X\leftarrow\\
&\wedge\\
\exists_{r_k \in P}:&X\in \left\{B_{1},\ldots,B_{n}\right\}
\end{align*}

The \cfp $P_{cf}\left(X_\iota\right)$ regarding $X$ and rule $r_k$ in program $P$ is the following transformation of
$P$:

\begin{enumerate}
\item Replace all non-negated literals $A$ in all rules with the auxiliary literal $A_{\iota}$;
\item Remove from $r_k$ the default literal $\sim X$;
\item Add to $P_{cf}\left(X_\iota\right)$ the rule $X\leftarrow X_{\iota}$
\end{enumerate}

The index used ($\iota$, iota) is unique for this \cfp and this particular rule and is called the \cf index of
$P_{cf}\left(X_\iota\right)$. It identifies both the rule used to try to conclude $X$ and the literal we are trying to
conclude. Hence the notation $P_{cf}\left(X_\iota\right)$.

\end{definition}




%########################################### DEFINITION
\begin{definition}[\CFP Set $\beta_{cf}$]

Let $P$ be a \nlp with rules in the form $r_k=H\leftarrow A_{1},\ldots,A_{m},\sim B_{1},\ldots,\sim B_{n}$ ($1\leq k\leq
l$) and $I=\left\{I_{1},\ldots,I_{j}\right\}$ a set of literals such that $1\leq i\leq j$. The set of all \cfps of $P$
regarding all literals in $I$ and all rules in $P$ is the set

\begin{center}
\begin{math}
\beta_{cf}^{I}\left(P\right) = \displaystyle\bigcup_{\iota} P_{cf}\left(X_{\iota}\right)
\end{math}
\end{center}

with $\iota$ being the program's \cf index, incremented as a new rule or literal is considered.

\end{definition}




%########################################### DEFINITION
\begin{definition}[Elegant \CFA Assumptions operator $\alpha_{cf}$]

Let $P$ be a \nlp, $I=\left\{I_{1},\ldots,I_{j}\right\}$ a set of literals such that $1\leq i\leq j$ and $B$ the \cfp
set obtained from the application of $\beta_{cf}^{I}\left(P\right)$.

For each \cfp $P_{cf}\left(X_{\iota}\right)$ regarding a literal $X_{\iota}$ that contains a rule in the form:

\begin{center}
\begin{math} H_{\iota}\leftarrow A_{1},\ldots,A_{m},\sim B_{1},\ldots,\sim B_{n}
\end{math}
\end{center}

such that 

\begin{align*}
\exists_{Y\in \left\{B_{1},\ldots,B_{n}\right\}}:P_{cf}\left(Y\right)&\in B\\
&\wedge\\
Y&\neq X
\end{align*}

introduce a copy of the rule regarding $H_{\iota}$ such that $\sim Y$ is replaced with $H_{\kappa}$, $\kappa$ being the
\cf index identifying $P_{cf}\left(Y\right)$. Add this rule to $P_{cf}\left(X_{\iota}\right)$. 

The application of $\alpha_{cf}^{I}\left(B\right)$ results in a program set with auxiliary assumptions added to each
rule in $B$.

\end{definition}




%########################################### DEFINITION
\begin{definition}[$\Gamma^{r}$ operator]

Let $P$ be a \nlp and $I$ an interpretation. The revised program $P^{r}$ is the following sequence of operations:

\begin{enumerate}
\item Remove from all rules in $P$ the default literal $\sim X$ such that $X$ belongs to the head of the rule;
\item Create the \cfp set $B^{r}=\beta_{cf}^{I}\left(P_{r}\right)$ from interpretation $I$ and the revised program
$P^{r}$;
\item Apply the \cfa assumptions operator $\alpha_{cf}^{I}\left(B^{r}\right)$ on $B^{r}$;
\item Let $P^{r}=P^{r}\cup  \alpha_{cf}^{B^{r}}\left(I\right)$.
\end{enumerate}

On $P^{r}$ apply the \go and remove all \cfa literals $X_{\iota}$ from the resulting set.

\end{definition}

I eventually wrote several pages based on this definition and tried all the examples we had imagined so far and all was
working well. Even though there were some tricks in between the results were promissing. I hand a hunch, however, that
our auxiliary assumptions could be too permissive, and I was right. Consider program

\begin{align*} P=\{&y\leftarrow\sim b\\
&b\leftarrow\sim c\\
&c\leftarrow\sim a\}
\end{align*}

whose only $rSM=\{y,c\}$. Let's create the CFPs for $b$ and $c$, the only literals which appear in the head of a rule
and negated in the body of other rules:

\begin{align*} 
P_cf\left(b\right)=\{b&\leftarrow b_1   & P_cf\left(c\right)=\{ c&\leftarrow c_2 \\
y_1&\leftarrow                                         & y_2&\leftarrow b_{2} \\
b_1&\leftarrow\sim c                               & b_2&\leftarrow\\
c_1&\leftarrow\sim a                               & c_2&\leftarrow\sim a \\
b_1&\leftarrow c_2\}                               & y_2&\leftarrow b_1\}
\end{align*}


The problem is that our CFPs allow us to prove that $\Gamma^{r}_{P}\left(\{b,c\}\right)=\{b,c\}$, 
even though not only $c$ wasn't in an OLON, as $b$ wasn't part of any kind of model (SM or rSM). $c$ is provable from 
he main program, as $a$ is not part of the model. And $b$ comes out from the auxiliary assumption $b_1\leftarrow c_2$.
So the problem was, again, with the auxiliary assumptions\ldots


\section{Houston, we have a problem\ldots}
\ldots was the subject of an email I sent to professor Moniz Pereira regarding this subject. The result, along with some
discussion with Alexandre Pinto was a somewhat different approach to the definition which had the following main
differences:
\begin{itemize}
\item Instead of generating only some CFPs, we generated all possible CFPs;
\item All literals in our CFPs' rules, including the default ones, were indexed;
\item The stable models of $P\cup P_{cf}s$ were the rSM of $P$. This involved trying to find out the right auxiliary
literals of our transformation, in order to give the intended SM;
\item The auxiliary assumptions were now defined differently: if $a_1$ depended on $\sim b_1$, we would add the
dependency for $b$'s counterfactual program, $a_1\leftarrow\sim b_1,\sim b_2$;
\item Additionally, the counterfactual assumption of the CFP would be replaced by an odd loop, in order to switch on or
off the odd loop.
\end{itemize}

Only to illustrate the definition, in the odd loop of 3 the CFPs would become:

\begin{align*} 
P_cf\left(a\right)=\{ a&\leftarrow a_1 & P_cf\left(b\right)=\{ b&\leftarrow b_2 & P_cf\left(c\right)=\{ c&\leftarrow
c_3\\
a_1&\leftarrow\sim b_{1},\sim b_{2} & a_2&\leftarrow b^{-} & a_3&\leftarrow\sim b_{3}\\
b_1&\leftarrow\sim c_{1} & b_2&\leftarrow\sim c_{2},\sim c_{3} & b_3&\leftarrow c^{-}\\
c_1&\leftarrow a^{-} & c_2&\leftarrow\sim a_{2} & c_3&\leftarrow\sim a_{3}, \sim a_{1}\\
a^{-}&\leftarrow\sim a^{+} & b^{-}&\leftarrow\sim b^{+} & c^{-}&\leftarrow\sim c^{+}\\
a^{+}&\leftarrow\sim a^{-}\} & b^{+}&\leftarrow\sim b^{-}\} & c^{+}&\leftarrow\sim c^{-}\}
\end{align*}

We can see the main differences: the auxiliary assumptions are now embebbed in the main rules of the program, all
literals have indexes and the odd loop can be switched on or off by considering wither $b^{-}$ or $b^{+}$ as being part
of the model. All examples appeared to be working well until the following program:

\begin{align*} P=\{&b\leftarrow\sim a,c\\
&c\leftarrow a\\
&a\leftarrow\sim b\}
\end{align*}

which has $rSM_1=\{a,c\}$ and $rSM_2=\{b\}$. 

\begin{align*} 
P_{cf}\left(a\right)=\{&a\leftarrow a_{1}\\
&b_{1}\leftarrow a^{-},c_{1}\\
&c_{1}\leftarrow a_{1}\\
&a_{1}\leftarrow\sim b_{1}\\
&a^{-}\leftarrow a^{+}\\
&a^{+}\leftarrow a^{-}\}
\end{align*}

\begin{align*} 
P_{cf}\left(b\right)=\{&b\leftarrow b_{2}\\
&b_{2}\leftarrow\sim a_{2},c_{2},\sim a_{1}\\
&c_{2}\leftarrow a_{2}\\
&a_{2}\leftarrow b^{-}\\
&b^{-}\leftarrow b^{+}\\
&b^{+}\leftarrow b^{-}\}
\end{align*}

\begin{align*} 
P_{cf}\left(c\right)=\{&c\leftarrow c_{3}\\
&b_{3}\leftarrow\sim a_{3},c_{3}\\
&c_{3}\leftarrow a_{3}\\
&a_{3}\leftarrow\sim b_{3}\\
&c^{-}\leftarrow c^{+}\\
&c^{+}\leftarrow c^{-}\}
\end{align*}


With our transformation, $b$ would never make it to the model because it would end up depending on both $a_2$ and $\sim
a_2$. The problem was that, even though the chain of support for $b$ demands $a$ to be true, in order to get to $\sim
b$, the chain also demanded $\sim a$ to be true. However, $\sim a$ would eventually become true as soon as we proved
that there was in fact a chain of support between $b$ and $\sim b$ and hence all the literals in the chain would have to
be false.

The solution we came up with was to replace all default literals $\sim X_1$ in the body of all rules by an extended
literal $X_1^{\ast}$, being that $X_1^{\ast}\leftarrow\sim X_1$ and $X_1^{\ast}\leftarrow\sim X$. This allowed us to,
while having $a_2$ in the stable model, therefore cutting one of the rules for $a_2^{\ast}$, keep the other one, because
$a$ was not part of the model.

It was an interesting idea, and a necessary one, but it was incomplete. With our well known odd loop of 3, we couldn't
get the models we wanted. A solution for this was discussed, which involved counting the number of negations we went
through as we were descending through the dependency graph for each literal (the problem was here\ldots), but these
ideas still weren't enough\footnote{I never wrote any formal definition of these ideas, since the Houston problem,
because all the problems appeared during testing of programs, no definition ever succeded in all programs}.

Here's why: not only did the auxiliary assumptions kept some odd loops active in the transformed program, as they
prevented some things from being proved. Let's look at the odd loop of 3. Our CFPs are\footnote{We are now only
generating the star rules for the even transitions over negation, starting from the literal we are trying to prove. this
was the very last changing I discussed with Professor Moniz Pereira.}:

\begin{align*} 
P_cf\left(a\right)=\{ a&\leftarrow a_1 & P_cf\left(b\right)=\{ b&\leftarrow b_2 & P_cf\left(c\right)=\{ c&\leftarrow
c_3\\
a_1&\leftarrow b_{1}^{\ast},\sim b_{2} & a_2&\leftarrow b^{-} & a_3&\leftarrow\sim b_{3}\\
b_1&\leftarrow\sim c_{1} & b_2&\leftarrow\sim c_{2}^{\ast},\sim c_{3} & b_3&\leftarrow c^{-}\\
c_1&\leftarrow a^{-} & c_2&\leftarrow\sim a_{2} & c_3&\leftarrow\sim a_{3}^{\ast}, \sim a_{1}\\
b_{1}^{\ast}&\leftarrow\sim b_{1} & c_{2}^{\ast}&\leftarrow\sim c_{2} & a_{3}^{\ast}&\leftarrow\sim a_{3}\\
b_{1}^{\ast}&\leftarrow\sim b & c_{2}^{\ast}&\leftarrow\sim c & a_{3}^{\ast}&\leftarrow\sim a\\
a^{-}&\leftarrow\sim a^{+} & b^{-}&\leftarrow\sim b^{+} & c^{-}&\leftarrow\sim c^{+}\\
a^{+}&\leftarrow\sim a^{-}\} & b^{+}&\leftarrow\sim b^{-}\} & c^{+}&\leftarrow\sim c^{-}\}
\end{align*}


Now, if $I=\{a,b\}$, $a_{1}$ must be part of the interpretation, because the only way to prove $a$ is through its CFP.
However, if $a_{1}$ is part of the interpretation, $c_{3}$ will be false, thus allowing $b_{2}$ to become true. But
$b_{2}$ can't be, or else the rule for $a_{1}$ will be cut. Even if we only generated the CFPs for the literals present
in the model, the problem still maintained: $b_{2}$ would depend solely on $c_{2}^{\ast}$ which, in order to become
false, would demand both $c$ and $c_{2}$ to be true. But $c$ can't be true, particularly if we are testing
$I=\{a,b\}$\ldots

The Houston definition was not ready yet\ldots



\section{Some initial conclusions} More or less at the same time I was realizing all the previous problems, I started
writing the several things I thought that contributed to the failure of all the previous definitions, as well as all the
things I liked/disliked. Here's a brief account on them:

\begin{description}
\item[How to generate CFPs] Two methods were tested to geenrate CFPs. The first one, used until the Houston problems,
used no indexes on the default literals. From the Houston definition onward the definition started using indexes on the
default negated literals. From these two methods, I prefer the approach of the Houston definition, i.e., generate
indexes on all literals. This is because it gives a better image of the reasoning we are trying to apply: our CFP is a
test program, to see if, by assuming a certain $\sim X$ we eventually get to $X$. ``get to'' is a way of saying that
there has to be a chain of support from the CF assumption to the literal we are trying to prove. It's easier to see this
chain if all literals are numbered. The rest of the method to generate CFPs is equal on all approaches. The CF
assumption is removed from the CFP and a rule is added to the CFP which connects it to the original program. This is a
garantee that this CFP only allows iits CF assumption to be concluded.

\item[When to generate CFPs] The idea of generating CFPs is uniquelly to allow some literal to be proven by RAA
reasoning. On some instances of the definition we generated alll possible CFPs possible, i.e., for all literals present
in the program. On other instances, we only generated CFPs for the literals present in the interpretation. This could be
constrained even further by demanding the literals for which we generated CFPs had to appear at least once in the head
of some rule, and at least once negated in the body of some rule. These two conditions are essential conditions for any
literal to be provable by RAA. At least these two conditions have to be satisfied.

\item[Auxiliary assumptions] How should we connect the several CFPs? Well, I really have no conclusion here. I can say
that the last approach was the one more intuitive for me, and the one that appeared to work the best. I believe that,
ideally, the connection between the CFPs should be implied from the original program, i.e., we shouldn't have to add any
rules to the CFPs. However, so far, no perfect way for doing this was defined\ldots

\item[Tricks] On the several tricks we've tested, I believe on the idea of not generating CFPs for facts. The reason is
simple: we want to generate as few CFPs as possible. So if something is a fact it is already proved. Additionally, I
also believe that all programs can be simplified by solving all the direct OLONs, although, as professor Moniz Pereira
said, this is only an optimization, it should not be mandatory in the definition. The trick of generating several CFPs,
for each CF assumption was obviously wrong. When we assume some $\sim A$, all $\sim A$s of the program are assumed, we
can't assume some and don't assume others.

\item[Using Stable Models on the transformed program] The use of stable models to see if a certain literal is provable
by RAA is an idea I don't find very intuitive, precisely because it doesn't reflect the original idea that an atom is
provable by RAA if there is a chain of support between the atom and its default negation. And this was probably the
thing that was annoying me the most in all our previous attempts.
\end{description}


This concludes the first part of this document. All the definitions we have tested so far, why they have failed and
where they have failed.



\newpage
\part{An alternative definition} Here I describe the last idea I had for a fixed point operator for revised stable
models, not based on CFPs.

\section{Origin of the idea} About the same time I wrote the previous text, Alexandre talked with me about  an idea he once had for an operator, but which he never explored. It basically consisted in inverting some rules of the program that
reflected the conditions the program was stating. For example in our odd loop of 3, we can see that the main condition for having $a$ is not to have $b$ and not to have $c$. One of the rules that states this is already present in the program ($a\leftarrow\sim b$), so apparently, we could add $a\leftarrow\sim c$ to the program. Now, if we proceed simmetrically to the other literals, our odd loop of 3 would become:

\begin{align*} 
P=\{&a\leftarrow\sim b & &b\leftarrow\sim a\\
&b\leftarrow\sim c & &c\leftarrow\sim b\\
&c\leftarrow\sim a & &a\leftarrow\sim c\}
\end{align*}

With this program, operator $\Gamma$ is able to return all three \rsms -- $\{a,b\},\{b,c\},\{c,a\}$. The idea was
interesting but the seemed to be no relevant principle behind it so I didn't thought of it for much longer. The principle I was missing came to me a couple of days later, when it was starting to become obvious to me that the ideas we had previously for our definition would not work. I then then considered the following: if we could get rid of the OLONs and ICONs, the \go would have no problem in calculating the models (revised or not). In other words, it would be handy if we could transform OLONs and ICONs into ELONs. How could we do that? For example, for the following program:

\begin{align*} 
P=\{&a\leftarrow b\\
&b\leftarrow c\\
&c\leftarrow\sim a\}
\end{align*}

the only thing we needed was to invert the last rule, and this would generate an ELON between $a$ and $c$:

\begin{align*} 
P=\{&a\leftarrow b\\
&b\leftarrow c\\
&c\leftarrow\sim a\\
&a\leftarrow\sim c\}
\end{align*}

Now, the only \sm of $P$ is $a$, which happens to be a \rsm. Curiosly, this was very close to the idea Alexandre discussed with me, although it was seen from a different pont of view. I showed him some preliminary results, looking promissing and we went on trying the idea on several other programs. Initially it was a rather \emph{na�ve} approach, we would invert every rule in the program, which had no justification possible. Obviously after a few programs the idea stopped working, and I put it to rest for a few days. I would return to it once more, only to try to understand why it didn't worked.


\section{H�agen-Dazs at Bel�m} The H�agen-Dazs shop near Torre de Bel�m, in Lisbon, is a great place to have breakfast
and ideas. It was there, on a Saturday morning, that I returned to my last suggestion of an operator for rSMs and rWFS,
using the previous intuitions and some new ideas I explore next.

I eventually returned to approach I had discussed with Alexandre because I thought that there had to be some reason for
it to work in some programs. I then did the oposite exercise (using the previous program): if I wanted to prove\footnote{I'm not sure if the notion of proof is the correct one, formally speaking\ldots} $a$,
what should be the body? Well, if we look at the loop, we can see that the only safe condition for $a$ to be true is for $c$ to be false. Why is the the only safe condition? Because, being the head of the last rule in the OLON's chain, if $a$ is to be true (a RSM), then the the head of the rule that depends on $\sim a$ must be false\footnote{In the case of this program, $b$ and $c$ will also be false but the safest condition is to use only the last rule in the chain. We'll see later why this is}. Now if $c$ if false, so will $b$, and so will $a$, by the first rule of the program. However, the last rule garantees, that $a$ is part of the \rsm. 

So what intuition have I followed: if an atom is in an OLON, then we can add the last rule of the OLON inverted. This rule solves the OLON, by transforming it into an ELON. There was, however, a catch: how do we know an atom is in an OLON, i.e., how do we know that we are able to use RAA reasoning in order to prove a certain literal? According to the definition of RSM\cite{ampMSc}:

\begin{quote}
	(\ldots) RAA reasoning takes place when $\sim a$ implies, directly or indirectly, $a$; and this happens only when tehre are OLONs in the program. In fact, that is the essential nature of an OLON: an atom depending on its own negation.
\end{quote}
 
This notion of \emph{dependency} can be grasped in many ways. Let's use here a sort of dependency graph, generated from the RAA assumption, for the previous program:

\begin{center}
\Tree
{
\K{$\sim a$}\B{d}\\
\K{$c$}\B{d}\\
\K{$b$}\B{d}\\
\K{\textbf{$a$}}
}
\end{center}

We can see that, just it was stated in the original definition, $\sim a$ implies, directly $a$. What about other programs? For example: 

\begin{align*} 
P=\{
&b\leftarrow\sim a,c\\
&c\leftarrow a\\
&a\leftarrow x\\
&x\leftarrow\sim b
\}
\end{align*}

which has $rSM_{1}=\{a,c\}$ and $rSM_{2}=\{b\}$. To begin with, we can see that $a$ is not in an OLON. It's easy to see
this if we draw the chain of support, starting with $\sim a$. 

\begin{center}
\Tree
{
\K{$\sim a$}\B{d}\\
\K{$b,[c]$}\B{d}\\
\K{$\sim x$}\B{d}\\
\K{$\sim a$}
}
\end{center}

I put $c$ in brackets to say that $c$ is a precondition to move on to the next branch of the tree. As we have reached an atom which is already present in the dependency graph, we stop the generation of branches. Additionally, we can see that by assuming $\sim a$ we won't reach $a$, so $a$ is not in an OLON. Well, if $a$ is not in an OLON, it must be proven by the classical notion of support, or it doesn't belong to any model (revised or otherwise). Either way, it is not important to our generation of rules. 

Moving to the other literals present in the program, $c$ never appears negated in the body of any
rule, so it can never be proven by RAA reasoning. Let's turn our attention now to $b$. 

\begin{center}
\Tree
{
\K{$\sim b$}\B{d}\\
\K{$x$}\B{d}\\
\K{$a$}\B{d}\\
\K{$c$}\B{d}\\
\K{$b,[\sim a]$}\\
}
\end{center}

$\sim b$ leads to $b$, meaning it is in an odd loop. It has a precondition that is $\sim a$ must be true. Aditionally, the head of the last rule in the chain must be false ($\sim x$). So if we test any interpretation which includes $b$, we must add the rule $b\leftarrow\sim x,\sim a$, where $\sim a$ is the precondition in brackets and $\sim x$ is the tail of the chain of support. 

So, our original program remains the same for all interpretations we test except for $I=\{b\}$. In this case we add the rule $b\leftarrow\sim x,\sim a$ to our program, which allows us to conclude that $I$ is a \rsm of $P$.

However, this idea of drawing the chain of support for each literal $X$, starting with $\sim X$ to see if we reach $X$ is not very elegant, from the mathematical point of view, so I put the intuition behind this in an operator. I called it $\Delta$ operator (\emph{delta}, which starts with ``d'', as the word ``dependency''). Intuitivelly it works somewhat like this:


\begin{definition}[$\Delta$ operator -- Intuitive definition]
Let $P$ be a NLP with rules $r$ in the form $r=H\leftarrow A_{1},\ldots,A_{m},\sim B_{1},\ldots,\sim B_{n}$, and $I$ and interpretation such that $I=\{L_{1},\ldots,L_{j}\}$. 

\begin{enumerate}
	\item $\Delta$ operator is to be applied to every literal $L_{j}$ in $I$ such that $L_{j}$ appears negated at least once in the tail of at least one rule of $P$, and $L_{j}$ appears at least once in the head of at least one rule of $P$.
	\item This being the case, which is a necessary condition for the existence of an OLON, \dop will generate an ordered set for each literal $L_{j}$, starting preciselly with each literal's default negation $\sim L_{j}$.
	\item For each of the ordered sets generated, if the ending element of the set is the negation of the starting element of the set, we generate a rule with the first two elements of the set, which will be something like $first\_element\leftarrow second\_element$.
	\item We then add the generated rules to the original program and can go on with the calculation of \go.
\end{enumerate}

\end{definition}


The main issues in this definition are:

\begin{itemize}
	\item The actual working of the \dop, i.e., specifically how do we go on generating each of the ordered sets;
	\item After generating the ordered sets, how do we \emph{transform} the ordered sets into rules.
\end{itemize}

So let's start with the the first point. The idea is that \dop starts with a certain $\sim X$, something that can only exist in the tail of a rule, and will eventually reach an $X$, something in the head of a rule. So, \dop will search for literals in the tails of rules and will return their heads, which is something like 

\begin{align*}
\Delta\left(\{X\}\right) = \{H\}, \exists_{r\in P}: r=H\leftarrow X.
\end{align*}

Now several things can happen when we search for literals in the tails of rules. One of them is that we don't have $X$ in the tail of a rule, but we do have $\sim X$. How should \dop work in this case? Well, it's easy to see that if we have a certain literal $X$ we are trying to prove, all rules that have $\sim X$ in the tail will have their heads false:

\begin{align*}
\Delta\left(\{X\}\right) = \{\sim H\}, \exists_{r\in P}: r=H\leftarrow\sim X.
\end{align*}

Something else that can happen is that we are actually looking the a negated literal. In this case, our rules would be the simetrical to the ones above:

\begin{align*}
\Delta\left(\{\sim X\}\right) = \{H\}, \exists_{r\in P}: r=H\leftarrow\sim X;\\
\Delta\left(\{\sim X\}\right) = \{\sim H\}, \exists_{r\in P}: r=H\leftarrow X.
\end{align*}


These four rules pretty much define the simplest case which is the existence of one (only one\ldots) rule in the program with precisely one literal in the body of the rule. But this case is rather unreal as most of the time we have rules with several literals in the body (conjunctions), and in some cases we also have several rules for the same literal (disjunctions). Now for the case of conjunctions, the idea is pretty simple: we just state what are the conditions we must obey in order to have the head of the rule. In other words, if $r=H\leftarrow W,X,Y,Z$ and we are calculating $\Delta\left(\{X\}\right)$, the result is $H$ with the condition that $\{W,Y,Z\}$ must hold. We are not interested in going down the trees of $W$,$Y$ and $Z$ because the operator is only concerned with showing if $X$ depends on some $\sim X$. Therefore, $\Delta\left(\{X,[W,Y,Z]\}\right)=\Delta\left(\{X\}\right)$ . If it does depend, then we are also concerned in knowing what other conditions must hold. So the conjunction rule would be something like:

\begin{align*}
\Delta\left(\{X\}\right) = \{H,[W_{1},\ldots,W_{n}]\}, \exists_{r\in P}: r=H\leftarrow X,W_{1},\ldots,W_{n}.
\end{align*}

Now this is the real general case, which we can extend to include the previous four rules:

\begin{align*}
\Delta\left(\{X\}\right) = \{H,[W_{1},\ldots,W_{n}]\}, \exists_{r\in P}: r=H\leftarrow X,W_{1},\ldots,W_{n}.\\
\Delta\left(\{X\}\right) = \{\sim H,[W_{1},\ldots,W_{n}]\}, \exists_{r\in P}: r=H\leftarrow \sim X,W_{1},\ldots,W_{n}.\\
\Delta\left(\{\sim X\}\right) = \{H,[W_{1},\ldots,W_{n}]\}, \exists_{r\in P}: r=H\leftarrow \sim X,W_{1},\ldots,W_{n}.\\
\Delta\left(\{\sim X\}\right) = \{\sim H,[W_{1},\ldots,W_{n}]\}, \exists_{r\in P}: r=H\leftarrow X,W_{1},\ldots,W_{n}.
\end{align*}

The only case missing to explore is when we have several rules sharing one same head tail $X$. What this case means is that by assuming $X$ we can prove several (possibly) different heads. So the idea is that we generate a copy of the ordered set as it is so far, and continue with both sets from that point:


\begin{align*}
\Delta\left(\{X\}\right) = \{H_{1},\ldots,H_{n}\}, \exists_{r_{1},\ldots,r_{n}\in P}:\\
r_{1}=H_{1}\leftarrow Body_{1},\ldots,r_{n}=H_{n}\leftarrow Body_{n};\\
\Delta\left(\{X_{1},\ldots,X_{n}\}\right) =\text{ create a copy of the ordered set}\\
\text{ as it is so far, and continue with the two instances separated.}
\end{align*}

We now have an idea of all the rules we can use when generating our dependency ordered sets. Now what rules dictate the generation of the rule that transforms the OLON into an ELON? Basically, we have to look at the generated ordered set to see the conditions that have to be respected. For example, $\Delta\left(\{\sim b\}\right)$ in the previous program would generate the following ordered set:

\begin{align*}
\{\{\sim b\},\{x\},\{a\},\{c\},\{b,[\sim a]\}\}
\end{align*}

From this set the rule we should generate the rule $b\leftarrow\sim x,\sim a$ which correspond the the idea we defined above, i.e., $\sim a$ is the precondition in brackets and $\sim x$ is the tail of the chain of support. However, more spicificity has to be added to these explanation. Consider a program with two RAA chains of support for the same literal over a conjunction:


\begin{align*} 
P=\{
&a\leftarrow x,y,\sim b\\
&x\leftarrow z\\
&z\leftarrow\sim a\\
&y\leftarrow w\\
&w\leftarrow\sim a
\}
\end{align*}

This program would have the following dependency graph

\begin{center}
\Tree
{
							& \K{$a$}\B{dl}\B{d}\B{dr}\\
\K{$x$}\B{d}	&	\K{$y$}\B{d} & \K{$\sim b$}\\
\K{$z$}\B{d}	& \K{$w$}\B{d}\\
\K{$\sim a$}	& \K{$\sim a$}\\}
\end{center}


and the application of \DOp would provide us with the following result:

\begin{align*}
\{\{\sim a\},\{w\},\{y\},\{a,[x,\sim b]\}\}\\
\{\{\sim a\},\{z\},\{x\},\{a,[y,\sim b]\}\}
\end{align*}

suggesting the rules we should add to the program would be:

\begin{align*}
a\leftarrow\sim w,x,\sim b\\
a\leftarrow\sim z,y,\sim b\\
\end{align*}

which is incorrect because both $x$ and $y$ will never be true in any rSM, as they are part of the RAA chain for each of the sides of the conjunction. The question here that the literals in the brackets are part of another RAA chain, and as such they will eventually be false. $\sim b$, however, is not part of any of the RAA chain so can be added as a constraint as usual. The other ones will be added negated:

\begin{align*}
a\leftarrow\sim w,\sim x,\sim b\\
a\leftarrow\sim z,\sim y,\sim b\\
\end{align*}

This allows us to define how to generate a rule, given an ordered set:

\begin{itemize}
	\item The head of the rule is the atom in the default literal in the first element of the ordered set;
	\item We add to the body of the rule, the negation of the second element of ordered set, if this atom is different from the atom in the first element\footnote{This allows it to work with the simplest OLON, $X\leftarrow\sim X$};
	\item Furthermore, we add all literals in brackets to the body of the rule in their original form, if they are not part of an RAA chain of the same literal, or negated, if they are part of the RAA chain of a literal.
\end{itemize}

We can now provide with a primary formal definition of the \DOp:





%%%%%%%%%%%%%%%%%%%%%% ACTUAL DEFINITION

\begin{definition}[$\Delta$ operator]
Let $P$ be a NLP with rules $r$ in the form $r=H\leftarrow A_{1},\ldots,A_{m},\sim B_{1},\ldots,\sim B_{n}$, and $I$ an interpretation such that $I=\{L_{1},\ldots,L_{j}\}$.

For all $L_{j}\in I$ such that

\begin{align*}
\exists_{r_{k}, r_{i}}: & r_{k} = H_{k}\leftarrow A_{k_{1}},\ldots,A_{k_{m}},\sim B_{k_{1}},\ldots,\sim B_{k_{n}} \wedge\\
& r_{i} = H_{i}\leftarrow A_{i_{1}},\ldots,A_{i_{m}},\sim B_{i_{1}},\ldots,\sim B_{i_{n}} \wedge\\
& L_{j} \in B_{i} \wedge L_{j} \in H_{k} (\text{with the possibility of } i=k).
\end{align*}

build the $L_{j}$'s dependency ordered set using the following rules:

\begin{align*}
\Delta\left(\{X\}\right) = \{H,[W_{1},\ldots,W_{n}]\}, \exists_{r\in P}: r=H\leftarrow X,W_{1},\ldots,W_{n};\\
\Delta\left(\{X\}\right) = \{\sim H,[W_{1},\ldots,W_{n}]\}, \exists_{r\in P}: r=H\leftarrow \sim X,W_{1},\ldots,W_{n};\\
\Delta\left(\{\sim X\}\right) = \{H,[W_{1},\ldots,W_{n}]\}, \exists_{r\in P}: r=H\leftarrow \sim X,W_{1},\ldots,W_{n};\\
\Delta\left(\{\sim X\}\right) = \{\sim H,[W_{1},\ldots,W_{n}]\}, \exists_{r\in P}: r=H\leftarrow X,W_{1},\ldots,W_{n};\\
\Delta\left(\{X,[W,Y,Z]\}\right)=\Delta\left(\{X\}\right);\\
\Delta\left(\{X\}\right) = \{H_{1},\ldots,H_{n}\}, \exists_{r_{1},\ldots,r_{n}\in P}:\\
r_{1}=H_{1}\leftarrow Body_{1},\ldots,r_{n}=H_{n}\leftarrow Body_{n};\\
\Delta\left(\{X_{1},\ldots,X_{n}\}\right) =\text{ create a copy of the ordered set}\\
\text{ as it is so far, and continue with the two instances separated}.
\end{align*}

Generate the rules regarding each dependency ordered set using the following rules:

\begin{itemize}
	\item The head of the rule is the atom in the default literal in the first element of the ordered set;
	\item We add to the body of the rule, the negation of the second element of ordered set, if this atom is different from the atom in the first element\footnote{This allows it to work with the simplest OLON, $X\leftarrow\sim X$};
	\item Furthermore, we add all literals in brackets to the body of the rule in their original form, if they are not part of an RAA chain of the same literal, or negated, if they are part of the RAA chain of a literal.
\end{itemize}

Add the generated rules to $P$.

\end{definition}

So, this operator generates ordered sets, which start with the default literal we are assuming to test our RAA chain.
Everytime we reach a point where we have a conjunction, we simply put the head of that conjunction in brackets. We
don't need to go down that conjunction because this operator's purpose is not to prove a certain literal -- its
objective is simply to show the literals we have to go through to reach a certain $X$ starting with $\sim X$. If, at
any point, we reach a disjunction, i.e., two rules, with the same literal in the tail, we generate two derivation
trees, which are equal until the disjunction point, and continue from that point onward.

I'll now show how this behaves in some programs, showing the result of the
$\Delta$ operator. These programs are generally enhanced versions of the programs available in the previous tests
set Alexandre gave me.

%%%%%%%%%%%%%%%%%%%%%%%%
\begin{align*} 
P_{1}=\{
&b\leftarrow y\\
&y\leftarrow \sim a,c\\
&c\leftarrow a\\
&a\leftarrow x\\
&x\leftarrow\sim b
\}
\end{align*}

This is an enhanced version os a previous program. $a$ and $b$ are in an ELON. Furthermore, $b$ is also  in an OLON through $c$, $a$ and $x$, although having a precondition of $\sim a$.


\begin{align*} 
\Delta\left(\{\sim a\}\right) = \{\{\sim a\},\{y,[c]\},\{b\},\{\sim x\},\{\sim a\}\}
\end{align*}

\DOp stops because it reaches a literal it already passed by. In this case, we started from $\sim a$ and ended up in $\sim a$, so $\sim a$ is not in an OLON. No rule is to be generated here.


\begin{align*} 
\Delta\left(\{\sim b\}\right) = \{\{\sim b\},\{x\},\{a\},\{c\},\{y,[\sim a]\},\{b\}\}
\end{align*}

We reached $b$ starting from $\sim b$, which means $b$ is in an OLON. It also has a precondition of $\sim a$ and as $\sim a$ is not part of any other RAA chain, it will be added without modifications. The generated rule is $b\leftarrow\sim x,\sim a$ and is only added when $b$ is aprt of the interpretation.

We can now see that the intended rSMs can be calculated as SMs with \GM:

\begin{align*} 
\Gamma\left(\{a,c,x\}\right) = \{a,c,x\}\\
\Gamma\left(\{b,y\}\right) = \{b,y\}
\end{align*}



%%%%%%%%%%%%%%%%%%%%%%%%
\begin{align*} 
P_{2}=\{
&a\leftarrow x\\
&x\leftarrow \sim a,y\\
&y\leftarrow \sim b\\
&b\leftarrow z\\
&z\leftarrow \sim b,w\\
&w\leftarrow\sim a
\}
\end{align*}


Both $a$ and $b$ are in OLONs through $x$ and $z$ respectively. however, both of them have preconditions ($\sim b$ and $\sim a$ respectively).

\begin{align*} 
\Delta\left(\{\sim b\}\right) = & \{\{\sim b\},\{y\},\{x,[\sim a]\},\{a\},\{\sim x\}\}\\
& \{\{\sim b\},\{y\},\{x,[\sim a]\},\{a\},\{\sim w\},\{\sim z\},\{\sim b\}\}\\
& \{\{\sim b\},\{z,[w]\},\{b\}\}\\
\\
\Delta\left(\{\sim a\}\right) = & \{\{\sim a\},\{w\},\{z,[\sim b]\},\{b\},\{\sim z\}\}\\
& \{\{\sim a\},\{w\},\{z,[\sim b]\},\{b\},\{\sim y\},\{\sim x\},\{\sim a\}\}\\
& \{\{\sim a\},\{x,[y]\},\{a\}\}\\
\end{align*}

The first two dependency ordered sets  don't lead to $b$. The last one leads and the to add if $b\leftarrow\sim z,w$. Simmetrically for $a$, the rule to add is $a\leftarrow\sim x,y$.

The rSMs are:

\begin{align*} 
\Gamma\left(\{a,y\}\right) = \{a,y\}\\
\Gamma\left(\{b,w\}\right) = \{b,w\}
\end{align*}




%%%%%%%%%%%%%%%%%%%%%%%%
%\begin{align*} 
%P_{3}=\{
%&a\leftarrow \sim b, \sim c\\
%&b\leftarrow \sim a, \sim c\\
%&c\leftarrow \sim b, \sim a
%\}
%\end{align*}
%
%
%%%%%%%%% NOT DONE THIS ONE; TOO MUCH WORK!!!!!
%
%
%Both $a$ and $b$ are in OLONs through $x$ and $z$ respectively. however, both of them have preconditions ($\sim b$ and $\sim a$ respectively).
%
%\begin{align*} 
%\Delta\left(\{\sim b\}\right) = & \{\{\sim b\},\{y\},\{x,[\sim a]\},\{a\},\{\sim x\}\}\\
%& \{\{\sim b\},\{y\},\{x,[\sim a]\},\{a\},\{\sim w\},\{\sim z\},\{\sim b\}\}\\
%& \{\{\sim b\},\{z,[w]\},\{b\}\}\\
%\\
%\Delta\left(\{\sim a\}\right) = & \{\{\sim a\},\{w\},\{z,[\sim b]\},\{b\},\{\sim z\}\}\\
%& \{\{\sim a\},\{w\},\{z,[\sim b]\},\{b\},\{\sim y\},\{\sim x\},\{\sim a\}\}\\
%& \{\{\sim a\},\{x,[y]\},\{a\}\}\\
%\end{align*}
%
%The first two dependency ordered sets  don't lead to $b$. The last one leads and the to add if $b\leftarrow\sim z,w$. Simmetrically for $a$, the rule to add is $a\leftarrow\sim x,y$.
%
%The rSMs are:
%
%\begin{align*} 
%\Gamma\left(\{a,y\}\right) = \{a,y\}\\
%\Gamma\left(\{b,w\}\right) = \{b,w\}
%\end{align*}
%
%





%%%%%%%%%%%%%%%%%%%%%%%%
\begin{align*} 
P_{4}=\{
&a\leftarrow x\\
&x\leftarrow y\\
&y\leftarrow \sim a\\
&y\leftarrow \sim b
\}
\end{align*}

$a$ is in an OLON but can be proven in two diferent ways.

\begin{align*} 
\Delta\left(\{\sim a\}\right) = \{\{\sim a\},\{y\},\{x\},\{a\}\}\\
\end{align*}

The rule to add is $a\leftarrow\sim y$. Note that the added rule clearly prevents $a$ from being proven by RAA if $y$ is in the model. This allows for a way of seeing if something was proven by RAA or using classical support.

The only rSM is:

\begin{align*} 
\Gamma\left(\{a,x,y\}\right) = \{a,x,y\}
\end{align*}







%%%%%%%%%%%%%%%%%%%%%%%%
\begin{align*} 
P_{5}=\{
&a\leftarrow x\\
&x\leftarrow y,\sim a\\
&y\leftarrow \sim a
\}
\end{align*}


$a$ is in two RAA chains.

\begin{align*} 
\Delta\left(\{\sim a\}\right) = & \{\{\sim a\},\{y\},\{x,[\sim a]\},\{a\}\}\\
& \{\{\sim a\},\{x,[y]\},\{a\}\}
\end{align*}

Note that $\sim a$ is actually provable by RAA through a conjunction, and both sides of the conjunction end up in $a$. So the rules to add are $a\leftarrow\sim y$ (where we don't include $\sim a$ as that is the literal we came from in the first place) and $a\leftarrow\sim x,\sim y$ (where $y$ is added negated because it is in another RAA chain for $\sim a$).

$a$ is the only rSM.

\begin{align*} 
\Gamma\left(\{a\}\right) = \{a\}
\end{align*}












%%%%%%%%%%%%%%%%%%%%%%%%
\begin{align*} 
P_{6}=\{
&a\leftarrow x\\
&x\leftarrow y,\sim x\\
&y\leftarrow \sim a
\}
\end{align*}

Both $a$ and $x$ are in OLONs with preconditions which deny each other.


\begin{align*} 
\Delta\left(\{\sim x\}\right) = & \{\{\sim x\},\{x,[y]\}\}\\
\Delta\left(\{\sim a\}\right) = & \{\{\sim a\},\{y\},\{x,[\sim x]\},\{a\}\}
\end{align*}

The generated rules are $x\leftarrow y$ and $a\leftarrow\sim y,\sim x$. $a$ is the only rSM, even though $x$ is also in a RAA chain.

\begin{align*} 
\Gamma\left(\{a\}\right) = \{a\}
\end{align*}







%%%%%%%%%%%%%%%%%%%%%%%%
\begin{align*} 
P_{7}=\{
&a\leftarrow \sim b\\
&b\leftarrow \sim c,x\\
&c\leftarrow \sim a\\
&x\leftarrow a, z\\
&z\leftarrow \sim x
\}
\end{align*}

This is the OLON of 3, with a few extra rules.

\begin{align*} 
\Delta\left(\{\sim a\}\right) = & \{\{\sim a\},\{c\},\{\sim b\},\{a\}\}\\
\Delta\left(\{\sim b\}\right) = & \{\{\sim b\},\{a\},\{\sim c\},\{b,[x]\}\}\\
                                & \{\{\sim b\},\{a\},\{x,[z]\},\{\sim z\},\{\sim x\}\}\\
                                & \{\{\sim b\},\{a\},\{x,[z]\},\{b,[\sim c]\}\}\\
\Delta\left(\{\sim c\}\right) = & \{\{\sim c\},\{c\},\{\sim b\},\{a\}\}\\
                                & \{\{\sim c\},\{c\},\{\sim b\},\{a\}\}\\
                                & \{\{\sim c\},\{c\},\{\sim b\},\{a\}\}\\
\Delta\left(\{\sim x\}\right) = & \{\{\sim a\},\{c\},\{\sim b\},\{a\}\}\\
\end{align*}

I show the generated rules side by side with the program, to be easier to check the several rSMs:
\begin{align*} 
P_{7}=\{&a\leftarrow \sim b     &  Rules_{a}=\{&a\leftarrow \sim c\}\\
        &b\leftarrow \sim c,x   &  Rules_{b}=\{&b\leftarrow \sim a,\sim x\\
        &c\leftarrow \sim a     &              &b\leftarrow \sim a,z,c\}\\
        &x\leftarrow a, z       &  Rules_{c}=\{&c\leftarrow \sim b,x\}\\
        &z\leftarrow \sim x\}		&  Rules_{x}=\{&x\leftarrow \sim z,a\}
\end{align*}


\begin{align*} 
\Gamma\left(\{a\}\right) = \{a\}
\end{align*}









%%%%%%%%%%%%%%%%%%%%%%%%
\begin{align*} 
P_{8}=\{
&a\leftarrow x,y\\
&x\leftarrow z\\
&z\leftarrow \sim a\\
&y\leftarrow w\\
&w\leftarrow \sim a
\}
\end{align*}



%%%%%%%%%%%%%%%%%%%%%%%%
\begin{align*} 
P_{9}=\{
&a\leftarrow \sim a, \sim b\\
&b\leftarrow \sim b,d\\
&d\leftarrow \sim a
\}
\end{align*}



%%%%%%%%%%%%%%%%%%%%%%%%
\begin{align*} 
P_{10}=\{
&a\leftarrow x\\
&x\leftarrow \sim a\\
&b\leftarrow \sim a\\
&c\leftarrow \sim b\\
&d\leftarrow \sim c
\}
\end{align*}




%%%%%%%%%%%%%%%%%%%%%%%%
\begin{align*} 
P_{11}=\{
&a\leftarrow \sim b\\
&b\leftarrow a
\}
\end{align*}



%%%%%%%%%%%%%%%%%%%%%%%%
\begin{align*} 
P_{12}=\{
&y\leftarrow \sim a\\
&a\leftarrow \sim b\\
&b\leftarrow \sim c
\}
\end{align*}




%%%%%%%%%%%%%%%%%%%%%%%%
\begin{align*} 
P_{13}=\{
&a\leftarrow \sim x,\sim y\\
&x\leftarrow \sim z\\
&z\leftarrow \sim a\\
&y\leftarrow \sim w\\
&w\leftarrow \sim a
\}
\end{align*}






%########################################### DEFINITION
%\begin{definition}[Prototype of the H�agen-Dazs definition] Let $P$ be a NLP and I a 2-valued interpretation, such that
%$I=\{a_1,\ldots,a_n\}$. Let $RAA$ be a subset of $I$, such that all literals in $RAA$ appear at least once in the head
%of a rule and at least once negated in the body of a rule.
%
%For each literal $a_n$ in $RAA$ generate a chain of support starting from the CF assumption $\sim a_n$. If this chain of
%support leads to $a_n$, the n$a_n$ can be proven by RAA. Add a rule reflecting that to $P$, following the next rules
%\footnote{TODO: Missing a condition that explains Fages program}:
%
%\begin{itemize}
%\item If the chain has no intermediate literals, there are no conditions. The rule to add is simply $a_n\leftarrow$.
%\item If the chain has intermediate literals, the negation of the head $H$ of the last rule in the chain is the
%condition to add: $a_n\leftarrow\sim H$.
%\item Additionally, if throughout the chain any conjunctions appear, the heads of those the other branches have to be
%considered. How do we consider them?
%
%\begin{itemize}
%\item If the branch also comes from $\sim a_n$, the head $B$ of that branch has to be false -- $a_n\leftarrow\sim H,\sim
%B$
%\item Otherwise, the head $B$ of the branch has to be true -- $a_n\leftarrow\sim H,B$
%\end{itemize}
%
%
%
%\end{itemize}
%
%These rules are added to $P$ forming $P^r$ and the SMs of $P^r$ are the rSMs of $P$.
%
%\end{definition}
%
%
%So, this definition has two things I particularly prefer to all the others: it only changes the program in those rules
%where we clearly show that literals are involved in OLONs and in order to show that, it uses precisely the same
%intuition we use when we are looking at the program. 
%
%I've tested the definition in all the programs that so far have given us problems, and it passed. I tested with the
%Fages program, and although it passed I still have to write down the intuition I followed (I only finished it a few
%hours ago). I also tested with the loop of 5 and the Gelfond program and it also passed.
%
%So far, the only problems I can think of are the explanation of the process for creating the rules and a way of
%generating the chain of support without having to draw the trees (so it becomes easier to make an implementation).





\bibliographystyle{abbrv}




%%%%% NO CITES...

%\nocite{nmrGrigoris}
%\nocite{milkBook}
%\nocite{aparasPhD}
%\nocite{ampMSc}
%\nocite{jjaPhD}


\bibliography{rsm}





\end{document}
