\documentclass[12pt]{article}
%%%%%%%%%%%%%%%%%%%%%%%%%%%%%%%%%%%%%%%%%%%%%%%%%%%%%%%%%%%%%%%%%%%%%%%%%%%%%%%%%%%%%%%%%%%%%%%%%%%%%%%%%%%%%%%%%%%%%%%%%%%%
\usepackage{graphicx}
\usepackage{amsmath}

%TCIDATA{OutputFilter=LATEX.DLL}
%TCIDATA{Created=Mon Oct 02 17:16:05 2000}
%TCIDATA{LastRevised=Mon Oct 02 17:28:23 2000}
%TCIDATA{<META NAME="GraphicsSave" CONTENT="32">}
%TCIDATA{<META NAME="DocumentShell" CONTENT="General\Blank Document">}
%TCIDATA{CSTFile=LaTeX article (bright).cst}

\newtheorem{theorem}{Theorem}
\newtheorem{acknowledgement}[theorem]{Acknowledgement}
\newtheorem{algorithm}[theorem]{Algorithm}
\newtheorem{axiom}[theorem]{Axiom}
\newtheorem{case}[theorem]{Case}
\newtheorem{claim}[theorem]{Claim}
\newtheorem{conclusion}[theorem]{Conclusion}
\newtheorem{condition}[theorem]{Condition}
\newtheorem{conjecture}[theorem]{Conjecture}
\newtheorem{corollary}[theorem]{Corollary}
\newtheorem{criterion}[theorem]{Criterion}
\newtheorem{definition}[theorem]{Definition}
\newtheorem{example}[theorem]{Example}
\newtheorem{exercise}[theorem]{Exercise}
\newtheorem{lemma}[theorem]{Lemma}
\newtheorem{notation}[theorem]{Notation}
\newtheorem{problem}[theorem]{Problem}
\newtheorem{proposition}[theorem]{Proposition}
\newtheorem{remark}[theorem]{Remark}
\newtheorem{solution}[theorem]{Solution}
\newtheorem{summary}[theorem]{Summary}
\newenvironment{proof}[1][Proof]{\textbf{#1.} }{\ \rule{0.5em}{0.5em}}


\begin{document}


\begin{center}
{\large The Axioms of Absolute Geometry -Part 1\bigskip }
\end{center}

\textbf{The Incidence Axioms}\smallskip 

\emph{Points},\emph{\ lines} and\emph{\ planes} are undefined terms.

\begin{description}
\item[I-1]  All lines and planes are sets of points.

\item[I-2]  Given any two points, there is exactly one line containing them

\item[I-3]  Given any three noncollinear points, there is exactly one plane
containing them.

\item[I-4]  If two points lie in a plane $E$, then the line containing the
two points lies in $E.$

\item[I-5]  If two planes intersect, their intersection is a line.

\item[I-6]  Every line contains at least two points and there are at least
three noncollinear points. Every plane contains at least three noncollinear
points and there are at least four noncoplanar points.
\end{description}

\textbf{The Betweenness Axioms\smallskip }

The notion of \emph{betweenness}, as in ``the point $B$ is between the
points $A$ and $C$'', is taken to be an undefined term. If a point $B$ is
between the points $A$ and $C$, we write $A-B-C$.

It is assume that betweenness satisifies the following five axioms:

\begin{description}
\item[B-1]  If $A-B-C$, then $C-B-A$. (If $B$ is between $A$ and $C$, then $B
$ is between $C$ and $A$.)

\item[B-2]  Given three collinear points $A,B$ and $C,$ then exactly one of
the following is true: $A-B-C$, $B-A-C$, or $A-C-B$. (Given any three
collinear points, exactly one point is between the other two.)
\end{description}

\begin{definition}
Let $A,B,C,$ and $D$ be four collinear points. We write $A-B-C-D$ provided
each of the following relations hold: $A-B-C$, $A-B-D$, $A-C-D$, and $B-C-D$.
\end{definition}

\begin{description}
\item[B-3]  Any four points of a line can be named in order $A,B,C,$and $D,$
in such a way that $A-B-C-D$.

\item[B-4]  If $A$ and $B$ are any two points, then there is a point $D$
such that $A-B-D$ and there is a point $C$ such that $A-C-B$.

\item[B-5]  If $A-B-C$, then $A,B,$ and $C$ are collinear.
\end{description}

We are now ready to introduce the definition of line segment, ray and angle.

\begin{definition}
(line segment) If $A$ and $B$ are two points, the line segment between $A$
and $B$ is the set points between $A$ and $B$ together with $A$ and $B$.
This line segment is usually denoted $\overline{AB}$.
\end{definition}

Note that $\overline{AB}=\left\{ C:A-C-B\right\} \cup \left\{ A\right\} \cup
\left\{ B\right\} $.

\begin{definition}
(ray) Let $A,B$ be two points. The\emph{\ ray with from }$A$\emph{\ through }%
$B$, denoted $\overrightarrow{AB}$, is the set of all points $C$ such that $%
C-A-B$ does not hold. The point $A$ is called the\emph{\ end point} of the
ray.
\end{definition}

\begin{definition}
(angle) An\emph{\ angle} is the union of two rays which have the same
endpoint but do not lie on the same line. If the angle is the union of the
rays $\overrightarrow{AB}$ and $\overrightarrow{AC}$, the angle is denoted
by $\angle BAC$. The rays $\overrightarrow{AB}$ and $\overrightarrow{AC}$
are called the \emph{sides of the angle} and the point $A$ is called the 
\emph{vertex of the angle}.
\end{definition}

Here are a couple of sample theorems which can be established using the
above axioms and definitions.

\begin{theorem}
If $A$ and $B$ are two points, then $\overline{AB}=\overline{BA}$.
\end{theorem}

\begin{theorem}
Let $\overrightarrow{AB}$ be a ray. If $C$ is a point on $\overrightarrow{AB}
$, then $\overrightarrow{AB}=\overrightarrow{AC}$.\medskip 
\end{theorem}

\textbf{The plane separation axiom\smallskip }

\begin{definition}
A set $K$ is said to be \emph{convex} provided that the line segment $%
\overline{AB}$ is contained in $K$ whenever the points $A$ and $B$ are in $K.
$
\end{definition}

\begin{description}
\item[PS-1]  (Plane separation axiom) Given a line and a plane containing
it, the set of all points of the plane that do not lie on the line is the
union of two disjoint sets $H_{1}$ and $H_{2}$ such that each of the sets is
convex and if $P$ belongs to one of the sets and $Q$ belongs to the other,
then the segment $\overline{PQ}$ intersects the line. Each of these sets is
called a\emph{\ half plane}.
\end{description}

\begin{definition}
Let $l$ be a line and let $A,B$ be points. If $A$ and $B$ are in a half
plane with edge $l$, then $A$ and $B$ are said to be on the same side of $B.$
If $E=H_{1}\cup l\cup H_{2}$, where $H_{1}$ and $H_{2}$ are the half planes
given by the plane separation axiom, then $H_{1}$ and $H_{2}$ are called
opposite sides of $l$.. Two points $A,B$ are said to be on opposite sides of 
$l$ if there is a plane $F$ be a plane containing $l$ , $A,$ and $B$, $%
F=G_{1}\cup l\cup G_{2}$ where $G_{1}$ and $G_{2}$ are half planes $A\in
G_{1}$ and $B\in G_{2}$
\end{definition}

\textbf{Angle Measurement Axioms\medskip }

We let $m(\angle ABC)$ denote\emph{\ the measure of the angle }$\angle ABC.$
It is assumed that the measure of an angle is a real number.

\begin{description}
\item[AM-1]  Given an angle $\angle A$, $m$ assigns one and only one real
number to $\angle A$.

\item[AM-2]  For every angle $\angle A$, $m\angle A$ is between $0$ and $180$%
.

\item[AM-3]  (Angle Construction Axiom) Let $\overrightarrow{AB}$ be a ray
on the edge of a half plane $H$. For every number $\alpha $ between $0$ and $%
180$ there is exactly one ray $\overrightarrow{AC}$ in $H$ such that $%
m\left( \angle BAC\right) =\alpha $.

\item[AM-4]  (Angle Addition Postulate). If $D$ is in the interior of $%
\angle ABC$, then $m\left( \angle ABC\right) =m\left( \angle ABD\right)
+m\left( \angle DBC\right) $.
\end{description}

Recall that two angles are \emph{supplementary} if the sum of their measures
is $180$.

\begin{description}
\item[AM-5]  (Supplement Axiom) If two angles form a linear pair, then they
are supplementary.
\end{description}

In a metric geometry, two angles are defined to be\emph{\ congruent} if they
have the same measure.

\begin{theorem}
Let $\angle ABC$ be any angle, $\overrightarrow{B^{\prime }C^{\prime }}$ a
ray and $H$ a half-plane with an edge that contains $\overrightarrow{%
B^{\prime }C^{\prime }}.$ then there is exactly one ray $\overrightarrow{%
B^{\prime }A^{\prime }}$, $A^{\prime }$ in $H,$ such that $\angle ABC\cong
\angle A^{\prime }B^{\prime }C^{\prime }.$.\medskip 
\end{theorem}

\begin{theorem}
If $D$ is in the interior of $\angle ABC$, $D^{\prime }$ is in the interior
of $\angle A^{\prime }B^{\prime }C^{\prime },$ $\angle ABD\cong \angle
A^{\prime }B^{\prime }D^{\prime }$ and $\angle DBC\cong \angle D^{\prime
}B^{\prime }C^{\prime },$ then $\angle ABC\cong \angle A^{\prime }B^{\prime
}C^{\prime }.\medskip $
\end{theorem}

\begin{definition}
If $D$ is in the interior of $\angle ABC$, $D^{\prime }$ is in the interior
of $\angle A^{\prime }B^{\prime }C^{\prime },$ $\angle ABD\cong \angle
A^{\prime }B^{\prime }D^{\prime }$ and $\angle ABC\cong \angle A^{\prime
}B^{\prime }C^{\prime },$ then $\angle DBC\cong \angle D^{\prime }B^{\prime
}C^{\prime }.$\smallskip 
\end{definition}

\begin{definition}
Let $\overrightarrow{AB}$ and $\overrightarrow{AC}$ be opposite rays and $D$
a point not on the line $\overleftrightarrow{CB}.$ Then $\angle BAD$ and $%
\angle DAC$ form a linear pair.\smallskip 
\end{definition}

\begin{definition}
If $m\left( \angle ABC\right) +m\left( \angle DEF\right) =$ $180,$ then $%
\angle ABC$ and $\angle ABC$ are called supplementary angles\smallskip 
\end{definition}

\begin{definition}
If the angles of a linear pair are congruent, then each of them is called a
right angle.\smallskip 
\end{definition}

\begin{definition}
Two rays are called perpendicular if their union is a right angle. An angle
is acute if its measure is less than $90^{\circ }$ and obtuse if their
measure is greater than $90^{\circ }.\smallskip $
\end{definition}

\begin{definition}
Two angles form a vertical pair if their sides form pairs of opposite rays.
Note that if $C^{\prime }-A-C$ and $B^{\prime }-A-B$, then $\angle BAC$ and $%
\angle B^{\prime }AC^{\prime }$ are a pair of vertical angles$.$
Alternatively, if $\overrightarrow{AC^{\prime }}$ and $\overrightarrow{AC}$
are opposite rays and $\overrightarrow{AB^{\prime }}$ and $\overrightarrow{AB%
}$ are opposite rays, then $\angle BAC$ and $\angle B^{\prime }AC^{\prime }$
are a pair of vertical angles. Moreover, any pair of vertical angles can be
labelled $\angle BAC$ and $\angle B^{\prime }AC^{\prime }$ in such a way as
to have $C^{\prime }-A-C$ and $B^{\prime }-A-B$
\end{definition}

Now for a rather famous theorem:\medskip

\begin{theorem}
(Vertical Angle Theorem) It two angles form a vertical pair, then they are
congruent.
\end{theorem}

\end{document}

