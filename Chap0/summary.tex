% =================================================================
%     Summary
% =================================================================
\selectlanguage{english}
\chapter*{Summary}

The \WFS(WFS) for \nlps associates with each program one single model expressing truth, falsity and undefinedness of atoms. Under the WFS, atoms are said to be undefined if:

\begin{itemize}
	\item Either are part of a two-valued choice (true in some worlds, false in others) but never undeniably true or false;
	\item Or are not classically supported;
	\item Or depend on an already undefined literal, as in one of the previous cases.
\end{itemize}

Undefinedness due to lack of classical support could be overcome by the introduction of another form of support, which would allow the WFS to deal with programs requiring other forms of reasoning besides the classical support, and thus gain expressiveness. One of these forms of support whose application has already been studied in the \RSMs semantics (rSMs) is support by \raa (RAA). This principle states that an hypothesis should be true if by assuming it's false this assumption leads to a contradition.

In this thesis we propose to study the application of the RAA principle to the WFS, thus defining the \RWFS (RWFS). Besides this definition we'll also study the definition of a fixed-point operator $\Gamma^{r}$, a counterpart of \GFo $\Gamma$, with support for RAA reasoning, and use this operator to perform the calculation of rSMs and the \rwfm of \nlps. We will also study a new property of rSMs and the definition of the \rpsms.

This thesis concludes with the discussion of several open issues and possible next research paths.



% =================================================================
%     Sumário
% =================================================================
%%\selectlanguage{portuges}
\chapter*{Sumário}

A Semântica Bem-Fundada (SBF) dos programas lógicos normais associa a cada programa um único modelo que expressa a verdade, falsidade e indefinição dos seus átomos. Os átomos, na SBF, dizem-se indefinidos se:

\begin{itemize}
	\item Ou são parte de uma escolha a dois valores (verdadeiros em alguns mundos, falsos noutros) mas nunca são inquestionavelmente verdadeiros ou falsos;
	\item Ou não têm suporte clássico;
	\item Ou dependem de átomos que já são indefinidos, por um dos casos anteriores.
\end{itemize}

A indefinição devida à falta de suporte clássico pode ser ultrapassada através da introdução de outra forma de suporte, o que permitiria à SBF lidar com programas lógicos normais de acordo com outras formas de raciocínio para além do suporte clássico, ganhando desta forma expressividade. Uma destas formas de raciocínio cuja aplicação foi já estudada na semântica dos Modelos Estáveis Revistos (MERs) é o suporte por \emph{redução ao absurdo} (RAA). Este princípio diz que uma hipótese deverá ser verdadeira se, por causa de se assumir que ela é falsa, essa assumpção nos leva a uma contradição.

Nesta tese propômo-nos a estudar a aplicação do princípio de RAA na SBF, definindo assim a Semântica Bem-Fundada Revista. Para além desta definição, vamos também estudar a definição de um operador de ponto fixo $\Gamma^{r}$, que será a versão com suporte por RAA do operador Gelfond-Lifschitz $\Gamma$, e vamos usar este operador para calcular os MERs e o modelo bem fundado revisto de programas lógicos normais. Vamos ainda estudar uma nova propriedade dos MERs e a definição dos modelos estáveis revistos parciais.

Esta tese culminará com uma discussão de vários pontos em aberto e do trabalho futuro a realizar neste contexto.


% =================================================================
%     Return to english as the main language
% =================================================================
\selectlanguage{english}

\incrementmtc
